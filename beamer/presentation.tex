\documentclass{beamer}

\mode<presentation>
{
  \usetheme{Warsaw}
  % or ...

  \setbeamercovered{transparent}
  % or whatever (possibly just delete it)
}

\usepackage{graphicx}
\usepackage{epstopdf}
\usepackage{listings}
\usepackage{hyperref}
\usepackage[english]{babel}
\lstset{language=Haskell}
% or whatever

\usepackage[latin1]{inputenc}
% or whatever

\usepackage{times}
\usepackage[T1]{fontenc}
% Or whatever. Note that the encoding and the font should match. If T1
% does not look nice, try deleting the line with the fontenc.


\title[Just Another Abstraction] % (optional, use only with long paper titles)
{Monad}

\author[Sibi] % (optional, use only with lots of authors)
{}
% - Use the \inst{?} command only if the authors have different
%   affiliation.

\institute[CHENSE Meetup] % (optional, but mostly needed)
{ 
  %% TCS Innovation Labs \\
  %% Cyber Physical Systems Lab
}
% - Use the \inst command only if there are several affiliations.
% - Keep it simple, no one is interested in your street address.

\date[] % (optional)
{March 1, 2014}
%% {\today}

\subject{Talks}
% This is only inserted into the PDF information catalog. Can be left
% out. 

% If you have a file called "university-logo-filename.xxx", where xxx
% is a graphic format that can be processed by latex or pdflatex,
% resp., then you can add a logo as follows:

% \pgfdeclareimage[height=0.5cm]{university-logo}{university-logo-filename}
% \logo{\pgfuseimage{university-logo}}



% Delete this, if you do not want the table of contents to pop up at
% the beginning of each subsection:
\AtBeginSubsection[]
{
  \begin{frame}<beamer>{Outline}
    \tableofcontents[currentsection,currentsubsection]
  \end{frame}
}


% If you wish to uncover everything in a step-wise fashion, uncomment
% the following command: 

%% \beamerdefaultoverlayspecification{<+->}


\begin{document}

\begin{frame}
  \titlepage
\end{frame}

\begin{frame}{About Me}
\begin{itemize}
\item
Sibi
\item
Haskell Programmer
\item
Internet Handle: psibi (github, Twitter)
\item
  Mail: sibi@psibi.in
\end{itemize}
\end{frame}

\begin{frame}{Outline}
  \tableofcontents
  % You might wish to add the option [pausesections]
\end{frame}


% Since this a solution template for a generic talk, very little can
% be said about how it should be structured. However, the talk length
% of between 15min and 45min and the theme suggest that you stick to
% the following rules:  

% - Exactly two or three sections (other than the summary).
% - At *most* three subsections per section.
% - Talk about 30s to 2min per frame. So there should be between about
%   15 and 30 frames, all told.

\section{Introduction}

\begin{frame}{Some basics}
\begin{itemize}

\item
Types
\item
TypeClasses
\item
Functions
\end{itemize}
\end{frame}

\begin{frame}{Maybe Type}
  % - A title should summarize the slide in an understandable fashion
  %   for anyone how does not follow everything on the slide itself.
  \begin{itemize}
  \item For values which may or may not have a value.
  \item data Maybe a = Just a | Nothing
  \item Examples:
    \begin{itemize}
    \item Just "hi"
    \item Just 3
    \item Nothing
    \end{itemize}
  \end{itemize}

\end{frame}

\section{Monad}

\begin{frame}{Definition}
  \begin{itemize}
  \item Category theory definition is scary!
  \item But you don't need to know that. :-)
  \item Best learnt with time and observing patterns.
  \end{itemize}
\end{frame}  

\begin{frame}[fragile]
\frametitle{Monad TypeClass}

\begin{lstlisting}
class  Monad m  where

  (>>=)            :: m a -> (a -> m b) -> m b

  return           :: a -> m a 
\end{lstlisting}
\end{frame}


\begin{frame}[fragile]{Maybe Monad}
\begin{lstlisting}
instance Monad Maybe where

    return x = Just x

    Nothing >>= f = Nothing  
    Just x >>= f  = f x
  
\end{lstlisting}
  
\end{frame}

\section{Sheep Cloning Experiment}

\begin{frame}[fragile]{Types and Functions}
  \begin{lstlisting}
type Sheep = ...

father :: Sheep -> Maybe Sheep
father = ...

mother :: Sheep -> Maybe Sheep
mother = ...
\end{lstlisting}
  
\end{frame}

\begin{frame}[fragile]{Other Functions}
\begin{lstlisting}
maternalGrandfather :: Sheep -> Maybe Sheep
maternalGrandfather s = case (mother s) of
                          Nothing -> Nothing
                          Just m  -> father m

mpg :: Sheep -> Maybe Sheep
mpg s = case (mother s) of
          Nothing -> Nothing
          Just m  -> case (father m) of
                       Nothing -> Nothing
                       Just gf -> father gf
\end{lstlisting}
\end{frame}

\begin{frame}[fragile]{Monads into action!}
\begin{lstlisting}
mpg :: Sheep -> Maybe Sheep
mpg s = (Just s) >>= mother >>= father >>= father 
\end{lstlisting}
\end{frame}

\section{Conclusion}
\begin{frame}{Conclusion}
  \begin{itemize}
  \item Monads are easy!
  \item Monads are fun!
  \item It's just another abstraction.
  \item There are lot of other abstractions. :-)
  \end{itemize}
\end{frame} 

\begin{frame}{Courtesy}
  \begin{itemize}
  \item All about Monads by Jeff Newbern

  \item Real World Haskell
  \end{itemize}
\end{frame}

\begin{frame}{Questions?}
  ???
\end{frame}

\end{document}
